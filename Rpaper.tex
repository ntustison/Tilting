%% This is file `elsarticle-template-1-num.tex',
%%
%% Copyright 2009 Elsevier Ltd
%%
%% This file is part of the 'Elsarticle Bundle'.
%% ---------------------------------------------
%%
%% It may be distributed under the conditions of the LaTeX Project Public
%% License, either version 1.2 of this license or (at your option) any
%% later version.  The latest version of this license is in
%%    http://www.latex-project.org/lppl.txt
%% and version 1.2 or later is part of all distributions of LaTeX
%% version 1999/12/01 or later.
%%
%% The list of all files belonging to the 'Elsarticle Bundle' is
%% given in the file `manifest.txt'.
%%
%% Template article for Elsevier's document class `elsarticle'
%% with numbered style bibliographic references
%%
%% $Id: elsarticle-template-1-num.tex 149 2009-10-08 05:01:15Z rishi $
%% $URL: http://lenova.river-valley.com/svn/elsbst/trunk/elsarticle-template-1-num.tex $
%%

%\documentclass[preprint,authoryear,review,12pt]{elsarticle}
\documentclass[final,5p,times,twocolumn]{elsarticle}

%% Use the option review to obtain double line spacing
%% \documentclass[preprint,review,12pt]{elsarticle}

%% Use the options 1p,two column; 3p; 3p,twocolumn; 5p; or 5p,twocolumn
%% for a journal layout:
%% \documentclass[final,1p,times]{elsarticle}
%% \documentclass[final,1p,times,twocolumn]{elsarticle}
%% \documentclass[final,3p,times]{elsarticle}
%% \documentclass[final,3p,times,twocolumn]{elsarticle}
%% \documentclass[final,5p,times]{elsarticle}
%% \documentclass[final,5p,times,twocolumn]{elsarticle}

\usepackage{epigraph}
\usepackage{color}
\usepackage{multirow,booktabs,ctable,array}
\usepackage{lscape}
\usepackage{amsmath}
\usepackage{lineno}
\usepackage{ulem}
\usepackage{setspace}
\usepackage{listings}
\usepackage{float}
\usepackage{listings}
\usepackage{color,colortbl}
\usepackage{rccol}
\usepackage[table]{xcolor}

    \definecolor{listcomment}{rgb}{0.0,0.5,0.0}
    \definecolor{listkeyword}{rgb}{0.0,0.0,0.5}
    \definecolor{listnumbers}{gray}{0.65}
    \definecolor{listlightgray}{gray}{0.955}
    \definecolor{listwhite}{gray}{1.0}
    \definecolor{lightcyan}{rgb}{0.88,1,1}

\definecolor{lightcyan}{rgb}{0,0,1}
\newcommand{\trEdit}[1]{{\color{blue}{#1}}}
\newcommand{\trDelete}[1]{{\color{blue}\sout{#1}}}

\floatstyle{plain}
\newfloat{command}{thp}{lop}
\floatname{command}{Command}

%\usepackage[nomarkers,notablist]{endfloat}

%% if you use PostScript figures in your article
%% use the graphics package for simple commands
%% \usepackage{graphics}
%% or use the graphicx package for more complicated commands
%% \usepackage{graphicx}
%% or use the epsfig package if you prefer to use the old commands
%% \usepackage{epsfig}

%% The amssymb package provides various useful mathematical symbols
\usepackage{amssymb}
%% The amsthm package provides extended theorem environments
% \usepackage{amsthm}
 
 \usepackage{makecell}

%% The lineno packages adds line numbers. Start line numbering with
%% \begin{linenumbers}, end it with \end{linenumbers}. Or switch it on
%% for the whole article with \linenumbers after \end{frontmatter}.
%% \usepackage{lineno}

%% natbib.sty is loaded by default. However, natbib options can be
%% provided with \biboptions{...} command. Following options are
%% valid:

%%   round  -  round parentheses are used (default)
%%   square -  square brackets are used   [option]
%%   curly  -  curly braces are used      {option}
%%   angle  -  angle brackets are used    <option>
%%   semicolon  -  multiple citations separated by semi-colon
%%   colon  - same as semicolon, an earlier confusion
%%   comma  -  separated by comma
%%   numbers-  selects numerical citations
%%   super  -  numerical citations as superscripts
%%   sort   -  sorts multiple citations according to order in ref. list
%%   sort&compress   -  like sort, but also compresses numerical citations
%%   compress - compresses without sorting
%%
%% \biboptions{comma,round}

% \biboptions{}

\providecommand{\OO}[1]{\operatorname{O}\bigl(#1\bigr)}

\graphicspath{
             {./Figures/}
             }

\long\def\symbolfootnote[#1]#2{\begingroup%
\def\thefootnote{\fnsymbol{footnote}}\footnote[#1]{#2}\endgroup}



\journal{NeuroImage}

\usepackage{Sweave}
\begin{document}
\input{Rpaper-concordance}


\begin{frontmatter}

\title{Tilting at Algorithmic Windmills}

\author[label1]{Nicholas J.~Tustison
  \fnref{label0}}
  \fntext[label0]{\scriptsize Corresponding author:  PO Box 801339, Charlottesville, VA 22908; T:  434-924-7730; email address:  ntustison@virginia.edu.
  }
\author[label2]{Brian B.~Avants}
\address[label1]{Department of Radiology and Medical Imaging, University of Virginia, Charlottesville, VA}
\address[label2]{Penn Image Computing and Science Laboratory, University of Pennsylvania,
                Philadelphia, PA}

%\maketitle

\linenumbers

\begin{abstract} 
Exploration of neuroscience hypotheses \trEdit{has} been  enhanced
by the increased availability of high-performance computational resources, large-scale
communal efforts such as the Insight Toolkit and the R statistical project, and
algorithmic advancements for data transformations and processing. 
An integral component of the vetting process for novel algorithmic techniques
includes comparison with other methods previously
established within the community.  Public availability, including 
open source distribution, of established packages has significantly 
facilitated these comparative evaluations.  However, complementing the
recent set of papers pointing to serious methodological and statistical
bias considerations in neuroimaging research, we point out potential 
issues associated with \trDelete{a type of} measurement bias in comparative algorithmic
evaluations and propose a set of guidelines for authors and reviewers
to minimize this confound.  We also illustrate these guidelines
through the use of a recent real-world scenario.
\end{abstract}
\begin{keyword}
open science \sep reproducibility  \sep \trEdit{software} \sep \trEdit{instrument bias}
%% keywords here, in the form: keyword \sep keyword
\end{keyword}

\end{frontmatter}
%
%
\newpage

%% MSC codes here, in the form: \MSC code \sep code
%% or \MSC[2008] code \sep code (2000 is the default)

%%
%% Start line numbering here if you want
%%
% \linenumbers

%% The Appendices part is started with the command \appendix;
%% appendix sections are then done as normal sections
%% \appendix

%% \section{}
%% \label{}

%% References
%%
%% Following citation commands can be used in the body text:
%% Usage of \cite is as follows:
%%   \citep{key}          ==>>  [#]
%%   \cite[chap. 2]{key} ==>>  [#, chap. 2]
%%   \citet{key}         ==>>  Author [#]

%% main text



\section{Introduction}
The neuroscience community has significantly benefitted from the proliferation
of imaging\trEdit{-related} software.  Established packages such as SPM \cite{Ashburner2012},
the FMRIB Software Library (FSL) \cite{Jenkinson2012}, and
the AFNI toolkit \cite{cox2012} have aided neuroimaging researchers 
around the world in performing complex analyses as part of 
ongoing neuroscience research.  Parallel to this line of
inquiry is continued algorithmic innovation for improvement
in analysis tools.  

As fellow scientists who actively participate in this 
research through our many contributions to the Insight
Toolkit \trEdit{[REF]} and other packages such as the Advanced Normalization Tools (ANTs),%
\footnote{
http://www.picsl.upenn.edu/ANTs
} 
we have noticed \trDelete{several}\trEdit{an increasing number of} publications in which a \trDelete{principle}\trEdit{principal}
component involves the comparison of algorithms\trEdit{, which in priciple is
  a good thing}.  \trDelete{One of our
concerns is}\trEdit{We are concerned, however, by} the lack of detail with which these comparisons are
often presented and the corresponding possibility of {\it instrumentation
bias} \cite{sackett1979} (considering software  as a type of 
instrument requiring proper ``calibration'' for accurate measurements).  
Based on our experience (including mistakes that we \trEdit{ourselves} have made), 
we propose an initial set of guidelines for minimal
reporting of algorithmic usage to minimize such bias, understanding that 
the discussion will hopefully 
elicit a much more comprehensive response from the larger community.

\trDelete{Although} \trEdit{Previous} articles have discussed similar concerns, oftentimes
within a \trDelete{much larger} \trEdit{specific application} context (e.g. fMRI reporting \cite{poldrack2008}), 
we are \trDelete{particularly}\trEdit{more generally} interested in comparative evaluations of software
and provide additional discussion through experience and actual 
examples found within the literature in addition to our own small,
yet important, real world scenario.
It is hoped that this commentary serves to raise awareness \trDelete{to}\trEdit{of} both 
authors and reviewers to such problematic issues (not unlike other
recent articles detailing additional potential sources of methodological 
bias \cite{kriegeskorte2009,vul2012,tustison2012}).


\section{Guidelines}
A solid evaluation is essential to a paper's longevity and
impact.  This requires effort in both planning the evaluation (before any
experiment is performed) and in publishing \trDelete{the evaluation}\trEdit{its details and results}.
\trDelete{We}\trEdit{Here, we} suggest general guidelines for both of these steps most of which
derive from basic scientific principles of clarity and reproducibility.

\subsection{Designing the Evaluation Study}


\subsubsection{Define the Algorithm's Context}
The only algorithm that is capable of outperforming all others on all
problems is exhaustive search\trEdit{, assuming a perfect objective function}.\trDelete{Therefore}\trEdit{As this is not feasible in practice}, many algorithms employ
prior knowledge such as regularization or other heuristics that
restrict the search space.  The very idea that one\trDelete{, for instance,
registration}\trEdit{(e.g., registration)} algorithm \trDelete{should be}\trEdit{could perform} better than all other algorithms on
all types of data is fundamentally flawed.  Indeed, the \trEdit{``}No Free Lunch
Theorems\trEdit{''} \trDelete{\cite{Wolpert1997}} provide bounds on solution quality that show otherwise
i.e. specifically that ``improvement of performance in problem-solving hinges on using prior information to match procedures to problems.''\trEdit{\cite{Wolpert1997}}


\subsubsection{Hassle the developers for a set of default parameters.}
Expert knowledge of a specific program and/or algorithm is most
likely found with the original developers who would be in a position to
provide optimal parameterization.
Developers should \trDelete{also}\trEdit{actively} facilitate
comparisons by providing default parameter files, sample scripts, 
or possible command line calls.  For example, the developers of
the image registration program elastix \cite{klein2010} provide an
assortment of parameter files on the wiki page%
\footnote{
  http://elastix.bigr.nl/wiki/index.php/Parameter\_file\_database
}
listed in tabular format complete with  short
description (including applied modality and object of interest) and 
any publications which used that specific parameter file.  Additionally,
other packages are associated with discussion forums where users can
query regarding usage.

\subsubsection{Do not \trDelete{code up}\trEdit{implement} your own version of the algorithm particularly 
if one is available from the original authors.  If you do, be sure to make
your implementation available.}

Collective experience in developing software has taught us the 
difficulty of translating descriptions found in
articles into functional code and that much is left unstated
in presenting core algorithmic ideas.  Ideally, the authors,
in conjunction with the manuscript, should provide an instantiation
of the code.  Unfortunately, as observed in \cite{kovacevic2006}
this is often not the case (even in terms of pseudocode).
As a result, comparative evaluations are sometimes carried out using 
code developed not by the original authors but rather the 
group doing the comparison.  This is not ideal as it \trEdit{is} questionable
whether or not the authors have sufficient motivation to ensure that
their instantiation of the algorithm is working correctly. For
example, in \cite{clarkson2011}, the authors compared three algorithms
for estimating cortical thickness.  Two of the algorithms were
coded by the authors (even though one was publicly available from 
the original authors although not advertised as such) while the third, 
Freesurfer \cite{fischl2012}, was used ``off the shelf'' having been extensively 
tuned and well-documented by the original authors.  
Thus, a natural question to ask is whether the performance difference
is due to the algorithm itself\trEdit{,} \trDelete{or} the implementation\trEdit{, or the parameter tuning}.


\subsubsection{Co-authors should verify findings.}

Although different journals have varying guidelines for determining 
co-authorship, there is at least an implied sense of responsibility for
an article's contents assumed by each of the co-authors.  Strategies 
taken by journal editorial boards are used to reduce undeserving
authorship attribution such as requiring the listing of the specific 
contributions of each co-author.  Additional proposals have included
signed statements of responsibility for the contents of the article 
\cite{nature2007}.  We suggest that at least one co-author
independently verify a subset of the results by running the processing
on their own computer.   The point of this excercise is to verify not only
reproducibility but also that the process can be explained in
sufficient detail.

\subsubsection{Comparisons should be performed on publicly available data.}
For reasons of reproducibility and transparency, evaluations should be performed
using publicly availabile data sets.  Given the rather large number
of such institutional efforts including 
IXI,%
\footnote{
http://www.brain-development.org
}
NKI,%
\footnote{
http://fcon\_1000.projects.nitrc.org/indi/pro/nki.html
}
OASIS,%
\footnote{
http://www.oasis-brains.org
}
Kirby,%
\footnote{
http://mri.kennedykrieger.org/databases.html
},
LONI,%
\footnote{
http://www.loni.ucla.edu/Research/Databases/
}, and
others, evaluations should include (if not be exhausted by)
comparisons using such data.  While evaluation on specific abnormal cohorts
might exclude such possibilities, such specific evaluations should
be extensively motivated in the introduction and/or discussion.
For example, if a particular algorithm with general application is 
found to perform better
on a private cohort of Parkinson's disease subject data, reasons
for performance disparity should be offered and supplemented with
analysis on public data.

\subsection{Publishing the Evaluation}

\subsubsection{Include parameters.}
A best practice exemplar of comparative evaluation studies
is the work by Klein et al. \cite{Klein2009} in which
14 nonlinear registration algorithms were compared using
four publicly available\trEdit{,} labeled brain \trEdit{MRI} data sets.  
As part of the study, the authors themselves tuned 
the parameters for their algorithm to ensure good
performance.  As
part of this work, Prof. Klein availed the scripts
used for the study on his website.%
\footnote{
http://mindboggle.info/papers/evaluation\_NeuroImage2009.php
}
This practice promotes reproducibility, fairness, and transparency in 
the evaluation and is perhaps most fundamental in these types
of studies for establishing conclusions regarding the relative
performance of algorithms.  

In contrast, not specifying parameters 
leaves one susceptible to criticisms of confirmation and/or 
instrumentation bias \cite{sackett1979}.  For example, in 
a recent paper \cite{haegelen2013} (similar issues can 
be found in \cite{luo2013,wu2013}), the authors compared
their ANIMAL registration algorithm with SyN \cite{avants2011} and determined
that ``[r]egistration with ANIMAL was better than with SyN for 
the left thalamus'' in a cohort of Parkinson's disease patients.
One of the stated reasons for using ANIMAL was ``because of 
[the authors']
extensive experience with this technique.''  However, for SyN,
the authors provide no equivalent assurance---no contact
was made with the primary developers of ANTs (SyN) nor
were any parameters listed for independent verification 
by reviewers or readers of the article.%
\footnote{
This is in addition to the authors' apparent neglect in  
removing the possible confound of using two different linear 
normalization strategies for the two algorithms.
}
Given the authors' relationship with one of the algorithms
and likely predisposition towards its relative performance,
particular caution has to be provided against possible confirmation
bias.  One of the easiest ways is to provide the set of parameters
used to run both algorithms so that others can ensure that 
the parameters selected are sufficient to warrant the conclusions
of the study.  Even better would be to include  the parameter
files or command line calls used in the study.


\subsubsection{Provide details as to the source of the algorithm}
As a corollary to the previous guideline, source of origin
should be provided for any code or package used during the
evaluation.  For example, N4 \cite{tustison2010} is a well-known 
bias correction algorithm first made available as a tech report
in the Insight Journal \cite{tustison2009a} for subsequent introduction
into the Insight Toolkit.  However, since
its inclusion, different instantiations have been made available.  
Apart from the source code that accompanied the original tech
report, N4 is also available in ANTs (the only version directly maintained
by the original authors).
It is also available as a module in Slicer,%
\footnote{
http://www.slicer.org/slicerWiki/index.php/Documentation/4.2/Modules/N4ITKBiasFieldCorrection
}
a wrapper of the \trEdit{Slicer} module in nipype,%
\footnote{
http://www.mit.edu/~satra/nipype-nightly/interfaces/generated/nipype.interfaces.slicer.filtering.n4itkbiasfieldcorrection.html
}
a module in \verb#c3d#,%
\footnote{
http://www.itksnap.org8/pmwiki/pmwiki.php?n=Convert3D.Documentation
}
and as a plugin in the BRAINS suite.%
\footnote{
http://www.nitrc.org/plugins/mwiki/index.php/brains:N4ITK
}
While each version is dependent on the original source
code, there could exist subtle variations which can affect performance.
For example, the parameters governing the spline distance and 
number of iterations directly affect outcome and should be tuned
based on application.  However, as a particular case, the \verb#c3d# 
implementation hard\trEdit{-}codes
these values with no access to modify them by the user, potentially
leading to suboptimal performance.

\subsubsection{Provide computational platform details of the evaluation}
A recent article \cite{gronenschild2012} pointed out significant
differences in \trEdit{FreeSurfer} output that varied with \trDelete{Freesurfer} release
version and with operating systems.  While the former is to be
expected given upgrades and bug fixes which occur between releases,
the latter underscores both the need for consistency in study processing
as well as the reporting of computational details for reproducibility. 

\subsubsection{Pre- and post-processing steps need to be supplied.}
In addition to disclosure of all parameters associated with the
methods to be compared, all processing steps from the raw 
to the final processed images in the workflow
need to be specified.    Justification for any deviation
of steps between algorithms needs to be provided.  Additionally,
failures during processing should be mentioned and whether
or not these data were excluded from subsequent analysis.  


\subsubsection{Resulting images should be publicly posted}

The current publishing paradigm limits the quantity of
results that can be posted.  There are only so many pages
allowed for a particular publication and displaying every
slice of every processed image, for example, is not feasible.  
This results in possible selection bias where results provided
in the manuscript are selected by the authors for demonstrating 
the effect postulated at the onset of the study.  Thus, differences
in performance assessment tend to be exaggerated based strictly
on visual representations in the paper.  Resources for data
sharing include MIDAS,%
\footnote{
http://www.midasplatform.org
}
XNAT,%
\footnote{
http://xnat.org
}
and slicedrop.%
\footnote{
http://slicedrop.com
}

%figshare

\subsubsection{Properly contextualize comparisons and motivate performance differences.}
In addition to algorithmic and study specifics, it is important to 
discuss potential limitations concerning qualitative and/or quantitative 
assessment metrics.  In \cite{rohlfing2012}, T. Rohlfing pointed out 
deficiencies in using standard overlap measures and image similarity
metrics in quantifying performance.  Other issues, such as biological
plausibility of the resulting transforms, need to also be considered.

Also important for inclusion is discussion on the possible reasons for performance
disparity.  If one algorithm outperforms another, reporting of those findings 
would be much more significant if the authors discussed possible reasons
for relative  performance levels.  This simply constitutes evidence for
the claims of the paper.

\subsection{Post-Publication: Publications Age}

quoting satra ``a published evaluation is old the day the evaluation was done, as most software keeps changing. for example, arno's results were fine at that time with those versions and parameterization of the software, but might be a different story currently.''

\section{Evaluation Example:  ANTS vs. antsRegistration}

\subsection{ANTS vs. antsRegistration:  Introduction}
As part of the recent Medical Image Computing and Computer-Aided Intervention 
(MICCAI) 2012 international meeting, several groups participated in the 
Multi-Atlas Labeling challenge%
\footnote{
https://masi.vuse.vanderbilt.edu/workshop2012
}
where participants proposed algorithmic workflows to propagate
labels from expertly annotated brains to testing data.  One of 
the overall conclusions drawn from the meeting was the significance 
of spatial normalization and so the organizers (Profs. Bennet Landman and Simon
Warfield) suggested rerunning the competition using a set
of transforms derived from a single registration algorithm.
Since three of the top four performing algorithms all used
the \verb#ANTS# registration tool \cite{avants2011} (available
in the Advanced Normalization Tools open source 
toolkit%
\footnote{
http://picsl.upenn.edu/ANTS
}) the organizers approached one of the principle authors of
the ANTs package (B.A.) to provide the ``canonical'' 
\verb#ANTS#-derived transforms.  
However, as part of the recent ITKv4 effort, a new
program for image registration, known as \verb#antsRegistration#
has been created and also released within ANTs \cite{avants2012}
where it continues to be refined (in contrast to \verb#ANTS# which
has been more or less stable for quite some time).

Motivated by this request and also to provide an example of the 
guidelines previously presented, we performed a comparison of 
 \verb#ANTS# and \verb#antsRegistration#.  The principle hypothesis
 to be explored with relevance to the challenge is:  {\it Does
 \verb#antsRegistration# outperform \verb#ANTS# in propagating
 labels for the MICCAI challenge training data?}  Important 
 corrollary hypotheses involve similar assessments of both the
 affine and deformable (i.e. SyN) components of the respective
 programs.
 
\subsection{ANTS vs. antsRegistration:  Materials and Methods}

\subsubsection{Images}
15 (10 females, 5 males) defaced T1-weighted MRI subjects 
were sampled from the 
publicly available 
Open Access Series of Imaging Studies (OASIS) 
project%
\footnote{
  http://www.oasis-brains.org
} 
which formed the ``training set'' for the challenge.  
Ages ranged between 19 and 34 (23 $\pm$ 4.3) years.
The complete
list of sampled subjects can be found in the challenge proceedings.
The corresponding labels
were provided by Neuromorphometrics, Inc.%
\footnote{
http://neuromorphemetrics.com/
} under academic subscription (released under the Creative Commons 
Atribution-Non Commercial license).

\subsubsection{Registration Protocols}
ANTs%
\footnote{
https://github.com/stnava/ANTs
}
is distributed using the github web-hosting service
\trDelete{facilitating}\trEdit{to facilitate} developer collaboration.  The version
of the ANTs toolkit that was used for this study 
was updated on Friday May 3, 2013.%
\footnote{
git tag: 2e907d081e21514d2edc736f8894d991f79cbab7 
}   
The computational platform for evaluation for both 
\verb#ANTS# and \verb#antsRegistration# is the Linux 
cluster managed by the University of Virginia for 
Computational Science and Engineering%
\footnote{
http://www.uvacse.virginia.edu
}
running the Portable Batch System (PBS)%
\footnote{
http://www.pbsworks.com
}
for queue management.  Each ``job'' consisted of a single
pairwise registration.  Since we performed pairwise registrations
for all possible pairings, this consisted of $15\times14=210$
registrations for each of the algorithms.  To send all 420 
jobs to the cluster
a perl script was created, \verb#runTiltingRegistrations.pl#,%
\footnote{
Note that a special github repository has been created for this
manuscript which includes all scripts for the evaluation.
}
which created a shell script for each pairwise registration.
Given a fixed and moving image pair denoted, respectively, as
\verb#$fixed# and \verb#$moving# and the ANTs path (\verb#$ANTsPATH#), 
the core command line calls
for each algorithm is given in Listing \ref{listing:ANTS} 
(\verb#ANTS#) and Listing \ref{listing:antsRegistration}
(\verb#antsRegistration#).  Note that the parameters were
constructed by two of the principle authors of ANTs 
(B.A. and N.T.) to maximize performance for both algorithms.

\lstset{frame = tb,
        framerule = 0.25pt,
        float,
        fontadjust,
        backgroundcolor={\color{listlightgray}},
        basicstyle = {\ttfamily\scriptsize},
        keywordstyle = {\ttfamily\color{listkeyword}\textbf},
        identifierstyle = {\ttfamily},
        commentstyle = {\ttfamily\color{listcomment}\textit},
        stringstyle = {\ttfamily},
        showstringspaces = false,
        showtabs = false,
        numbers = none,
        numbersep = 6pt,
        numberstyle={\ttfamily\color{listnumbers}},
        tabsize = 2,
        language=,
        floatplacement=!h,
        caption={\small \baselineskip 12pt ANTS command line call used for the evaluation. The help menu can be invoked by the command `ANTS --help'.},
        captionpos=b,
        label=listing:ANTS
        }
\begin{lstlisting}
${ANTsPATH}/ImageMath 3 $fixedTruncated \
  TruncateImageIntensity $fixed 0.01 0.99
${ANTsPATH}/ImageMath 3 $movingTruncated \
  TruncateImageIntensity $moving 0.01 0.99
  
${ANTsPATH}/ANTS 3 -o ${outputPrefix} \
  -m CC[${fixedTruncated},${movingTruncated},1,4] \
  -t SyN[0.1] -r Gauss[3,0] -i 100x100x70x20 \
  --number-of-affine-iterations 0 \
  --use-Histogram-Matching
\end{lstlisting}

\lstset{frame = tb,
        framerule = 0.25pt,
        float,
        fontadjust,
        backgroundcolor={\color{listlightgray}},
        basicstyle = {\ttfamily\scriptsize},
        keywordstyle = {\ttfamily\color{listkeyword}\textbf},
        identifierstyle = {\ttfamily},
        commentstyle = {\ttfamily\color{listcomment}\textit},
        stringstyle = {\ttfamily},
        showstringspaces = false,
        showtabs = false,
        numbers = none,
        numbersep = 6pt,
        numberstyle={\ttfamily\color{listnumbers}},
        tabsize = 2,
        language=,
        floatplacement=!h,
        caption={\small \baselineskip 12pt antsRegistration command line call used for the evaluation.  The help menu can be invoked by the command `antsRegistration --help'.},
        captionpos=b,
        label=listing:antsRegistration
        }
\begin{lstlisting}  
${ANTsPATH}/antsRegistration -d 3 -o ${outputPrefix} \
  -u 1 -w [0.01,0.99], -r [${fixed},${moving},1] \
  -t Rigid[0.1] -c [1000x500x250x100,1e-8,10] \
  -s 4x2x1x0 -f 8x4x2x1 \
  -m MI[${fixed},${moving},1,32,Regular,0.25] \
  -t Affine[0.1] -c [1000x500x250x100,1e-8,10] \
  -s 4x2x1x0 -f 8x4x2x1 \
  -m MI[${fixed},${moving},1,32,Regular,0.25] \
  -t SyN[0.1,3,0] -c [100x100x70x20,1e-9,15] \
  -s 3x2x1x0 -f 6x4x2x1 \
  -m CC[${fixed},${moving},1,4]
\end{lstlisting}

Following registration, the output was used
to warp the corresponding labels of the moving
image to the fixed image using \verb#antsApplyTransforms#:

\lstset{frame = tb,
        framerule = 0.25pt,
        float,
        fontadjust,
        backgroundcolor={\color{listlightgray}},
        basicstyle = {\ttfamily\scriptsize},
        keywordstyle = {\ttfamily\color{listkeyword}\textbf},
        identifierstyle = {\ttfamily},
        commentstyle = {\ttfamily\color{listcomment}\textit},
        stringstyle = {\ttfamily},
        showstringspaces = false,
        showtabs = false,
        numbers = none,
        numbersep = 6pt,
        numberstyle={\ttfamily\color{listnumbers}},
        tabsize = 2,
        language=,
        floatplacement=!h,
        caption={\small \baselineskip 12pt antsApplyTransforms command line call used to warp
        the moving label images to the fixed image for subsequent label overlap assessment}.,
        captionpos=b,
        label=listing:antsApplyTransforms
        }
\begin{lstlisting}  
${ANTsPATH}/antsApplyTransforms -d 3 \
  -i $movingLabels -r $fixed \
  -n NearestNeighbor \
  -o ${outputPrefix}LabelsWarped.nii.gz \
  -t $outputWarp -t $outputAffine
\end{lstlisting}
Label overlap assessment for each registration was quantified
using the \verb#LabelOverlapMeasures# available through ANTs and 
implemented in ITK \cite{tustison2009}.
It should noted that only the affine transform files are used
to assess affine algorithmic performance. In order to evaluate
the SyN component in isolation from the affine portion, the 
affine transforms from the better-performing algorithm (strictly
in terms of affine performance) were used as input for the other
algorithm which was then rerun for all data sets. 

\subsection{ANTS vs. antsRegistration:  Results}

Upon completion of all pairwise registrations, the labels for the
moving image were warped to the fixed image using \verb#antsApplyTransforms#
as described in Listing \ref{listing:antsApplyTransforms} using 
nearest neighbor interpolation.  This was performed 
for both affine and affine + deformable scenarios.  The
corresponding box plots are given in Figure \ref{fig:results}. 

\begin{center}
\begin{figure}
  \includegraphics[width=85mm]{tilting.pdf}
  \caption{Combined Dice values for all labels of the evaluation between the two ANTs-based registration programs including affine-only transformations.
  }
  \label{fig:results}
\end{figure}
\end{center}

Pairwise Students t-tests \trEdit{with Bonferroni correction} were performed for each of the three
comparisons\trDelete{ with Bonferroni correction}.
For the affine-only Dice value comparison, \verb#antsRegistration# 
only slightly \trDelete{output} \trEdit{out}performed \verb#ANTS# ($q = 0.053$, mean of the 
differences = 0.004).  For the comprehensive affine + SyN comparison
\verb#antsRegistration# yielded higher Dice values ($q = 0.053$, 
mean of the differences = 0.009).
Due to the difference in affine-only outcome, the \verb#ANTS# 
SyN deformable component was run again with initialization from
the affine output of the \verb#antsRegistration# runs.  This
latter set is denoted in Figure \ref{fig:results} as ``ANTS with
other affine.''  This scenario also showed \verb#antsRegistration# SyN 
producing slightly higher values than its ANTs counterpart 
($q = 0.053$, mean of the differences = 0.006).


%\begin{table}
%\begin{tabular*}{0.45\textwidth}{@{\extracolsep{\fill}}lr}
%  Comparison & q-value \\
%  \toprule
%  ANTS vs. antsRegistration (affine only) & 0.053 \\
%  ANTS vs. antsRegistration (affine + SyN) & 2.7e-14 \\
%  ANTS vs. antsRegistration (affine' + SyN) & 3.4e-13 \\
%  \bottomrule
%\end{tabular*}
%\end{table}

\subsection{ANTS vs. antsRegistration:  Discussion}
Strictly in terms of the Dice overlap metric, \verb#antsRegistration# yields
improved results over \verb#ANTS#.  Although the \trEdit{abstract algorithms implemented by the different} SyN \trEdit{software} components are theoretically
the same, \trEdit{implementation with} the newer program was constrained to conform
with the rigorous ITK coding standards and the plug-in architecture of
the generic registration framework resulting in additional possible
configurations not in \verb#ANTS#. Additionally, the newer program benefitted
from ITK's crowdsourcing of the contributed code (a great example of Linus's Law
in action---"with enough eyeballs, all bugs are shallow").  In contrast to the
SyN component, the affine components between the two programs differ substantially.
The parameters governing the affine registration in \verb#ANTS# were specifically 
tailored for brain registration and much of it is hard-coded internally within the
code.  In contrast, with \verb#antsRegistration# we took a much more general approach
by permitting the user to build up any number of linear ``registration stages'' 
to produce a final generic affine transform.

Other differences to be considered include the use of real types
\trEdit{(e.g., single vs. double precision, 80-bit x87 vs. 64-bit IEEE~754  computation)} between the
two programs and the use of multi-threading.  As regards the former, due to 
the constraints of certain  internals of ITK,  \verb#antsRegistration#
is currently implemented using \verb#double# instead of \verb#float#.  This
results in a larger memory footprint in addition to timing between the two
programs (single-threaded on a single core on the UVA cluster \verb#antsRegistration#
can run 3-4 times as long as \verb#ANTS#).  However, multi-threading capabilities
are much better developed in \verb#antsRegistration#. 

\trEdit{[Q: would it be worthwile showing the variability of results with
  single- vs. double-preecision ANTS? Also, does result change with number of
  CPUs? That would be interesting to demonstrate also. Finally, using standard
  vs. SSE double precision?]}

\section{Conclusion}

\epigraph{By honest I don't mean that you only tell what's true. But you make clear the 
entire situation. You make clear all the information that is required for somebody else 
who is intelligent to make up their mind.}{\trDelete{Professor }Richard Feynman}
\trEdit{[I think the Feynman quote would be much more appropriately positioned
  at the very beginning of the article, right after the abstract]}

Considering the permeation of the computational sciences in doing 
nueroimaging research, certain safeguards should be in place to prevent
(or at least minimize) potential biases and errors that can
unknowingly effect study outcomes.  There is no vetting agency for
ensuring that analysis programs used for research are reasonably error-free.
In addition, these software packages are simply ``black
boxes'' to many researchers who are not formally trained to debug
code.  And even to those of us who are, debugging other people's code
is oftentimes very difficult.  To this end, algorithmic comparisons are 
a very good way of evaluating general performance.  It is hoped that the
proposed set of guidelines set forth in this editorial aid the community
for future comparative assessments.


%% References with bibTeX database:

%\section*{Acknowledgments}



%\section{Notes/Thoughts}
%\begin{itemize}
%  \item Several kinds of bias have cropped up in the neuroimaging
%  community recently:
%  \begin{itemize}
%    \item{methodological bias}:  Kreikesgorte
%    \item{statistical bias}:  Vul2009, Tustison2012
%    \item  What about the dead salmon fmri study---multiple comparisons correction?
%    \item The single subject VBM study (assuming the single subject is representative of the population).
%    \item{overview}:  9 circles of scientific hell ---neuroskeptic %http://blogs.discovermagazine.com/neuroskeptic/?p=1205#.UYZxyJWO7Tx
%    \item Types of measurement bias: \cite{sackett1979}
%    \begin{itemize}
%      \item Instrument bias (code as a type of instrument which needs to be calibrated correctly).
%      \item Expectation bias (i.e. confirmation bias)
%    \end{itemize}  
%  \end{itemize}
%  We are concerned with the latter two.
%  \item How does one avoid confirmation bias?  If I'm comparing my
%  algorithm to somebody else's algorithm, I'm going to be naturally
%  predisposed to confirmatory evidence that my algorithm is better
%  while neglecting (most likely not with intent of doing so) evidence
%  that disconfirms my original belief (that my algorithm is better).
%  Richard Feynman quote (Cargo Cult Science):  
%  \begin{quote} 
%  The first principle [of science] is that you must not fool yourself?--and you are the easiest person to fool. So you have to be very careful about that. After you've not fooled yourself, it's easy not to fool other scientists. You just have to be honest in a conventional way after that.
%  \end{quote}
%  \item In ?The Meaning of It All? Feynman said, ``By honest I don?t mean that you only tell what?s true. But you make clear the entire situation. You make clear all the information that is required for somebody else who is intelligent to make up their mind.''
%  \item One particular aspect of the previous item is that it needs 
%  to be highlighted if somebody codes up the algorithm
%  to be compared themselves.  Writing bug-free code is difficult.  Look
%  at how many bug fixes are provided by the ITK community on a daily
%  basis.  
%  \item Also, is simply saying that one algorithm is better than 
%  another sufficient?  Should there be some theoretical explanation
%  for it.  For example, when I wrote my DMFFD image registration paper,
%  the comparison wasn't IRTK vs. DMFFD but rather FFD vs. DMFFD and
%  there were theoretical reasons why performance should be better
%  with the latter.  Also, I insisted to the reviewer on not using IRTK 
%  for comparison since there are also so many other performance-related
%  issues (type of interpolation, gradient step, metric, metric implementation, etc.) which would confound
%  comparisons.  Also, it probably helped that the theme had a more specific,
%  more verifiable focus, i.e. DMFFD produces a more efficient energy minimizer
%  since it acts as a preconditioner in the standard gradient descent
%  optimization.
%  \item Comparisons are best made with publicly available data and
%  results of the comparisons should be made available.
%  \item Operating system needs to be defined.  Freesurfer results
%  varied with MacOSx.
%  \item Ideally, the authors of the new algorithm would work
%  with the authors of the compared algorithm.  Given human
%  nature, this type of cooperation is not guaranteed.  However,
%  a minimum set of information is needed.
%  \item Arno Klein as an example of somebody who did it right.  Takes a
%  lot of work.
%  \item 
%  \item The source of the package software needs to be defined.  For 
%  example, N4 has been
%  instantiated in Slicer, ANTs, and c3d (also might be implemented
%  in one of Styner's NITRC projects).  However, each might use
%  different default parameters and have other tweaks which effects
%  performance.  In Vladimir Fonov's github repository containing 
%  various processing scripts for MINC, one can see from the history
%  how N4 was used but then the users switched back to N3MNI (due to 
%  performance issues?).  However, they used N4 out of the c3d package
%  which the original authors (N.T./B.A.) haven't touched in three years
%  so the parameters aren't optimal (shrink factors = 4, spline distance
%  = 100 with 3 levels: $100\times50\times50$).  Changing these parameters 
%  (which are crucial to performance) isn't
%  accessible to the user from c3d.  
%
% \item Collaborators should help verify findings by repeating a
%(representative subset of a) study independently, from scratch,
%preferably on another computer perhaps even with a different operating
%system.  A scientist can do this him/herself as well.  Personal
%experience suggests that this procedure will often uncover coding
%errors sometimes simply by forcing one to reread code or scripts again
%or clarify documentation.
% 
%  \item
%    \textcolor{blue}%{\href{http://en.wikipedia.org/wiki/Accuracy_and_precision}{Accuracy and precision (link)}} 
%        wiki link--- most studies ignore precision.  it
%    is critical in longitudinal studies.   precision is impacted by
%    parameter choices as well as computational architecture
%    (Freesurfer paper)
%
%   \item parameters impact accuracy - e.g. clarkson's paper vs our
%     recent tuning of DiReCT parameters w.r.t. expected thickness values.
%
%  \item  Biological plausibility is important---a more accurate method
%    may produce less biologically plausible results.
%    Cross-referencing with other fields is critical e.g. patterns of
%    MRI atrophy should match pathology distribution.  There is an art
%    to this, of course.
%
%  \item Prediction---an algorithm that yields a ``more significant''
%    result is not always better.  might mean ``finds more effected
%    areas'' or smaller p-values.  Testing prediction complements
%    studies of significance and cumulative distribution functions.
%
%  \item Verification is supported by open science technology for
%    sharing data and code.
%
%   \item Great resources for data-sharing:
%  	\begin{itemize}
%	   \item figshare
%           \item MIDAS 
%           \item slicedrop
%           \item R 
%        \end{itemize}
%
%  \item Great resources for code-sharing
%  	\begin{itemize}
%		  \item github
%		  \item NITRC
%		  \item ITK
%		\end{itemize}
%\end{itemize}

\section*{References}

\bibliographystyle{elsarticle-harv}
\bibliography{references}


%% Authors are advised to submit their bibtex database files. They are
%% requested to list a bibtex style file in the manuscript if they do
%% not want to use model1-num-names.bst.

%% References without bibTeX database:

% \begin{thebibliography}{00}

%% \bibitem must have the following form:
%%   \bibitem{key}...
%%

% \bibitem{}

% \end{thebibliography}


\end{document}

%%
%% End of file `elsarticle-template-1-num.tex'.
